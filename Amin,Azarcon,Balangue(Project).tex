\documentclass[a4paper,12pt]{article}
\usepackage{amsmath, amssymb, amsfonts}
\usepackage{graphicx}
\usepackage{hyperref}
\usepackage{physics}
\usepackage{mathtools}

\title{Symmetry Protection of Edge States in a 2D Topological Insulator}
\author{Amin Marie Ann, Azarcon Vince,Balangue Jabbar}
\date{\today}

\begin{document}

\maketitle

\begin{abstract}
  Investigate how time-reversal symmetry protects the gapless edge states in a 2D $Z_2$ topological insulator (like the Bernevig-Hughes-Zhang (BHZ) model).
\end{abstract}

\section{Mathematical Derivation}
\begin{enumerate}
    \item Start with the effective four-band BHZ model Hamiltonian in 2D:
    
    \[
    H(\mathbf{k}) = \begin{pmatrix}
    m(\mathbf{k}) & A k_x & 0 & -A k_y \\
    A k_x & -m(\mathbf{k}) & -A k_y & 0 \\
    0 & -A k_y & -m(\mathbf{k}) & -A k_x \\
    -A k_y & 0 & -A k_x & m(\mathbf{k})
    \end{pmatrix},
    \]
    where $m(\mathbf{k}) = M - B(k_x^2 + k_y^2)$.\\
Answer: $4 \times4 $  Hamiltonian in momentum space $k_\pm = (k_x + k_y)$ is written in a block diagonal form\\
\[ H(\mathbf{k}) = \begin{pmatrix}
h(\mathbf{k}) & 0\\
0 & h^\ast(\mathbf{k})
\end{pmatrix}\]
where $h(\mathbf{k}) $ is spin up and $h^\ast(\mathbf{-k})$ is spin down related by time reversal  symmetry.\\
The model is defined as
$\vert E, \uparrow \rangle, \vert H, \uparrow \rangle, \vert E, \downarrow \rangle, \vert H, \downarrow \rangle$\\
For spin up block $h(\mathbf{k}):$\\
This block couple: $\vert E, \uparrow \rangle, \vert H, \uparrow \rangle\\$\\

\[ h(\mathbf{k}) = \begin{pmatrix}
 m(\mathbf{k}) & A k_- \\
 A k_+ & -m(\mathbf{k} \\
\end{pmatrix}\]
where: $k\pm = k_x \pm k_y$ then substitute this to $k_- $ and $k_+$. We will get\\
\[ h(\mathbf{k}) = \begin{pmatrix}
 m(\mathbf{k}) & A (k_x - k_y)\\
 A (k_x + k_y) & -m(\mathbf{k}) \\
\end{pmatrix}\]\\
For spin downt this block couple: $\vert E, \downarrow \rangle, \vert H, \downarrow \rangle$\\
\[ h\ast(\mathbf{-k}) = \begin{pmatrix}
 -m(\mathbf{k}) & -A k_+\\
 -A k_- & m(\mathbf{k}) \\
\end{pmatrix}\]\\
Substitute  $k\pm = k_x \pm k_y$ to $k_- $ and $k_+$. We will get\\
\[ h(\mathbf{k}) = \begin{pmatrix}
 -m(\mathbf{k}) & -A (k_x - k_y)\\
 -A (k_x + k_y) & m(\mathbf{k}) \\
\end{pmatrix}\]\\
The given matrix for Hamiltonian is\\
    \[
    H(\mathbf{k}) = \begin{pmatrix}
    m(\mathbf{k}) & A k_x & 0 & -A k_y \\
    A k_x & -m(\mathbf{k}) & -A k_y & 0 \\
    0 & -A k_y & -m(\mathbf{k}) & -A k_x \\
    -A k_y & 0 & -A k_x & m(\mathbf{k})
    \end{pmatrix},
    \]\\
We can write this in spin system matrix\\
    \[
    H(\mathbf{k}) = \begin{pmatrix}
    m(\mathbf{k}) & A k_- & 0 & 0 \\
    A k_+ & -m(\mathbf{k}) & 0 & 0 \\
    0 & 0 & -m(\mathbf{k}) & -A k_+ \\
  0 & 0 & -A k_- & m(\mathbf{k})
    \end{pmatrix},
    \]\\
Why is that the $-A k_y$ in the diagonal becomes 0?\\
$\ast$ The $-A k_y$ does not becomes zero. Since BHZ model does not include spin-flip terms. Primarily due to no coupling between spin-up and spin-down in the BHZ model.

    \item Identify the time-reversal symmetry operator $\Theta$ for this system: $\Theta = \tau_0 \otimes i\sigma_y K$, where $K$ is complex conjugation.\\
Answer: The time reversal operator\\
\[
\Theta = \tau_0\otimes({i\sigma_y})K
\]\\
Where:\\
$\bullet \tau_0: 2\times2$ identity matrix in orbital space\\
$\bullet \sigma_y:$ acts in spin space\\
$\bullet{K}$: Complex conjugation\\
The BHZ Hamiltonian uses the basis\\
\[\vert E, \uparrow \rangle, \vert H, \uparrow \rangle, \vert E, \downarrow \rangle, \vert H, \downarrow \rangle\]\\
Time Reversal Symmetry implies
\[\Theta H({k})\Theta^{-1} = H({-k})
\]\\
We know that the $2\times2$ BHZ Hamiltonian diagonal block is\\
\[ H(\mathbf{k}) = \begin{pmatrix}
h(\mathbf{k}) & 0\\
0 & h^\ast(\mathbf{k})
\end{pmatrix}\]
Where\\
\[ h(\mathbf{k}) = \begin{pmatrix}
 m(\mathbf{k}) & A k_- \\
 A k_+ & -m(\mathbf{k} \\
\end{pmatrix}\]
For spin-up and $h^{\ast}(-k)$ is its time reversed counterpart
Let's apply $\Theta$ and $\Theta^{-1}$ to H(k)\\
\[ \Theta H(k)\Theta^{-1} = (\tau_0\otimes({i\sigma_y}))K (H(k)) K^{-1} (\tau_0\otimes(-{i\sigma_y}) )
\]
\[= (\tau_0\otimes({i\sigma_y})) H^{\ast}(k) (\tau_0\otimes(-{i\sigma_y}))\\ \]
Take the complex conjugate\\
\[ H^{\ast}(\mathbf{k}) = \begin{pmatrix}
h^{\ast}(\mathbf{k}) & 0\\
0 & h(-\mathbf{k})
\end{pmatrix}\]
Then:\\
\[\Theta H^{\ast}(k)\Theta^{-1} = \begin{pmatrix}
\sigma_{y} h^{\ast}(\mathbf{k}) \sigma_{y} & 0\\
0 & \sigma_{y }h(-\mathbf{k}) \sigma_{y}
\end{pmatrix}\]
Using this identity\\
\[\sigma_{y}( d \cdot \sigma)^{\ast} \sigma_{y} = - d \cdot \sigma
\]
Thefore\\
\[\sigma_{y}( h^{\ast}(k)) \sigma_{y} = h(-k)
\]
\[\sigma_{y}( h(-k)) \sigma_{y} = h^{\ast}(-k)
\]So\\
\[ \Theta H(k)\Theta^{-1} = \begin{pmatrix}
h^{\ast}(\mathbf{k}) & 0\\
0 & h(-\mathbf{k})
\end{pmatrix}\]
\[\Theta H(k)\Theta^{-1}= H(-k)\]
Thus, the BHZ Hamiltonian is invariant under time reversal.

    \item Analyze the bulk band structure and demonstrate the topological nature of the insulating phase by calculating the $Z_2$ topological invariant.\\
Answer: The bulk Hamiltonian in the BHZ model(in momentume space) is:\\
\[ H(\mathbf{k}) = \begin{pmatrix}
h(\mathbf{k}) & 0\\
0 & h^\ast(\mathbf{-k})
\end{pmatrix}\]
Each $h(k)$ is a $2\times2$:\\
\[h(k)=d(k)\cdot\sigma\]
\[h(k)=d_{x}(k)\cdot\sigma_{x} + d_{y}(k)\cdot\sigma_{y}+ d_{z}(k)\cdot\sigma_{z}\]
Where:\\
\[d_{x}(k) = Asink_{x}\]
\[d_{y}(k) = Asink_{y}\]
\[d_{x}(k) = M-B(cos k_{x} + cosk_{y}) \]
$\bullet A$: coupling parameter\\
$\bullet M$: mass term\\
$\bullet B$: band inversion control\\
Note: This model assumes a square lattice with lattice constant $a= 1$\\
For Bulk Band Structure\\
To find the energy bands diagonalize $h{k}$\\
\[E(k) = \pm|d(k)|\]
\[E(k) = \pm\sqrt{A^{2}(sin^{2}k_{x} + sin^{2}k_{y}) + \lbrack M - B(cos k_{x} + cos k_{y})\rbrack}\]
This gives two bands per spin sector. The bandsare gapped as long as $E(k) \neq 0$ for all $k$ in the brillouin zone\\ 

For the time reversal invariant systems in 2d. The $Z_{2}$ invariant can be computed from parity eigenvalues at the time reversal invariant momenta or TRIM\\
The TRIM points in 2D are:\\
$\Gamma = (0,0)$\\
$X = (\pi,0)$\\
$Y = (0,\pi)$\\
$M = (\pi,\pi)$\\

Fu and Kane's formula for inversion symmetry system is

\[(-1)^{\nu} = \prod_{i=1}^{4} \delta_{i}
\]
Where:
\[\delta_{1} = \prod_{n \in \text{occ}} \xi_{2n}(\Lambda_i)
\]
Where $\xi_{2n}(\Lambda_{i})$ is the parity eigenvalue of the $2n^{th}$ occupied Kramers pair at TRIM $\Lambda_{i}$\\
For BHZ model, These values depend on the sign of the mass term at each TRIM:\\
\[d_{z}(k) = M- B(cos k_{x} + cos k_{y})\]
$\bullet at (0,0): d_{z} = M - 2B$\\
$\bullet at (\pi,0),(0,\pi): d_{z} = M$\\
$\bullet at (\pi,\pi): d_{z} = M + 2B$\\
For Topological Phase Condition\\
$\bullet If M/B < 0:$Trivial Insulator\\
$\bullet If 0 < M/B < 2:$ Non Trivial Topological Insulator\\
Therefore, when M/B in the range $0 < M/B < 2$, the $Z_{2}$ invariant is \\
\[\nu = 1\]
indicating a quantum spin Hall Insulator


    \item Consider a finite-size strip geometry with open boundary conditions.\\
Answer: We now consider the BHZ model on a finite strip geometry that is infinite in the \( x \)-direction and finite in the \( y \)-direction. This models a 2D topological insulator with physical edges at \( y = 0 \) and \( y = L \), and imposes open boundary conditions in the \( y \)-direction:
\[
\Psi(0) = \Psi(L) = 0.
\]

The BHZ Hamiltonian in momentum space is given by:
\[
H(\mathbf{k}) =
\begin{pmatrix}
    m(\mathbf{k}) & A k_x & 0 & -A k_y \\
    A k_x & -m(\mathbf{k}) & -A k_y & 0 \\
    0 & -A k_y & -m(\mathbf{k}) & -A k_x \\
    -A k_y & 0 & -A k_x & m(\mathbf{k})
\end{pmatrix},
\]
where the mass term is:
\[
m(\mathbf{k}) = M - B(k_x^2 + k_y^2).
\]

Due to spin conservation, the Hamiltonian is block diagonal and can be decomposed into two \( 2 \times 2 \) blocks. Focusing on the spin-up block, we write the Hamiltonian in real space by replacing \( k_y \to -i \partial_y \):
\[
H_\uparrow(k_x, -i \partial_y) =
\begin{pmatrix}
M - B(k_x^2 - \partial_y^2) & A(k_x - i \partial_y) \\
A(k_x + i \partial_y) & -M + B(k_x^2 - \partial_y^2)
\end{pmatrix}.
\]

This results in a one-dimensional eigenvalue problem along the \( y \)-direction:
\[
H_\uparrow(k_x, -i\partial_y) \Psi(y) = E \Psi(y),
\]
where \( k_x \) acts as a good quantum number due to translational symmetry in the \( x \)-direction.

The goal is to find solutions \( \Psi(y) \) that are localized near the boundaries and lie within the bulk energy gap. These are the topologically protected edge states.
\begin{figure}[h]
\centering
\includegraphics[width = 0.5\linewidth]{Picture1.png}
\caption{Strip Geometry}
\label{fig.Picture1}
\end{figure}

    \item Analytically derive the dispersion relation of the edge states that appear within the bulk band gap.\\
Answer: We now derive the dispersion relation for the edge states that appear within the bulk energy gap of the BHZ model in a strip geometry. We focus on one spin block of the BHZ Hamiltonian, which in the continuum limit reads:
\[
H(k_x, k_y) = A(k_x \sigma_x + k_y \sigma_y) + m(k) \sigma_z,
\]
with mass term:
\[
m(k) = M - B(k_x^2 + k_y^2).
\]

In the strip geometry, \( x \) remains translationally invariant, while \( y \) is finite with open boundary conditions. We thus replace \( k_y \rightarrow -i\partial_y \), leading to:
\[
H(k_x, -i\partial_y) = A(k_x \sigma_x - i \partial_y \sigma_y) + [M - B(k_x^2 - \partial_y^2)] \sigma_z.
\]

We seek edge-localized solutions using the ansatz:
\[
\Psi(y) = e^{\lambda y} \chi, \quad \text{with } \Re(\lambda) < 0.
\]
Applying the Hamiltonian to this wavefunction yields the eigenvalue equation:
\[
\left[ A(k_x \sigma_x - i \lambda \sigma_y) + (M - B(k_x^2 - \lambda^2)) \sigma_z \right] \chi = E \chi.
\]

This can be written explicitly as:
\[
H_\lambda(k_x) =
\begin{pmatrix}
m_\lambda & A(k_x - \lambda) \\
A(k_x + \lambda) & -m_\lambda
\end{pmatrix}, \quad \text{where } m_\lambda = M - B(k_x^2 - \lambda^2).
\]

The eigenvalue equation leads to:
\[
E^2 = A^2(k_x^2 - \lambda^2) + m_\lambda^2.
\]

To obtain edge states, we consider a superposition of two exponentially decaying solutions:
\[
\Psi(y) = c_1 e^{\lambda_1 y} \chi_1 + c_2 e^{\lambda_2 y} \chi_2,
\]
with \( \Re(\lambda_{1,2}) < 0 \), and impose the boundary condition \( \Psi(0) = 0 \), which requires \( \chi_1 \) and \( \chi_2 \) to be linearly independent.

Edge states exist when \( E \) lies in the bulk band gap. For small \( k_x \), this yields a linear dispersion:
\[
E(k_x) = \pm A k_x.
\]

This describes helical edge states that propagate in opposite directions for opposite spins. These modes are topologically protected and remain gapless as long as time-reversal symmetry is preserved.
    \item Mathematically show how time-reversal symmetry enforces the Kramers degeneracy of the edge states at any momentum $k$ and prevents backscattering between counter-propagating edge states with opposite spin, thus protecting their gaplessness.\\
Answer: We now derive the dispersion relation for the edge states that appear within the bulk energy gap of the BHZ model in a strip geometry. We focus on one spin block of the BHZ Hamiltonian, which in the continuum limit reads:
\[
H(k_x, k_y) = A(k_x \sigma_x + k_y \sigma_y) + m(k) \sigma_z,
\]
with mass term:
\[
m(k) = M - B(k_x^2 + k_y^2).
\]

In the strip geometry, \( x \) remains translationally invariant, while \( y \) is finite with open boundary conditions. We thus replace \( k_y \rightarrow -i\partial_y \), leading to:
\[
H(k_x, -i\partial_y) = A(k_x \sigma_x - i \partial_y \sigma_y) + [M - B(k_x^2 - \partial_y^2)] \sigma_z.
\]

We seek edge-localized solutions using the ansatz:
\[
\Psi(y) = e^{\lambda y} \chi, \quad \text{with } \Re(\lambda) < 0.
\]
Applying the Hamiltonian to this wavefunction yields the eigenvalue equation:
\[
\left[ A(k_x \sigma_x - i \lambda \sigma_y) + (M - B(k_x^2 - \lambda^2)) \sigma_z \right] \chi = E \chi.
\]

This can be written explicitly as:
\[
H_\lambda(k_x) =
\begin{pmatrix}
m_\lambda & A(k_x - \lambda) \\
A(k_x + \lambda) & -m_\lambda
\end{pmatrix}, \quad \text{where } m_\lambda = M - B(k_x^2 - \lambda^2).
\]

The eigenvalue equation leads to:
\[
E^2 = A^2(k_x^2 - \lambda^2) + m_\lambda^2.
\]

To obtain edge states, we consider a superposition of two exponentially decaying solutions:
\[
\Psi(y) = c_1 e^{\lambda_1 y} \chi_1 + c_2 e^{\lambda_2 y} \chi_2,
\]
with \( \Re(\lambda_{1,2}) < 0 \), and impose the boundary condition \( \Psi(0) = 0 \), which requires \( \chi_1 \) and \( \chi_2 \) to be linearly independent.

Edge states exist when \( E \) lies in the bulk band gap. For small \( k_x \), this yields a linear dispersion:
\[
E(k_x) = \pm A k_x.
\]

This describes helical edge states that propagate in opposite directions for opposite spins. These modes are topologically protected and remain gapless as long as time-reversal symmetry is preserved.
\begin{figure}[h]
\centering
\includegraphics[width = 0.9\linewidth]{Picture2.png}
\caption{Protection of Edge State by Time Reversal Symmetry}
\label{fig;Picture2}
\end{figure}
\end{enumerate}

\end{document}
